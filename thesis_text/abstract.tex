\abstract{
  Nowadays the amount of raw textual data potentially containing knowledge in various fields is extremely large. The necessity of abstract formal knowledge led to the task of Information Extraction and \textit{Relation Extraction} as its part. This can be accomplished manually with the help of the experts in each of the considered domains, but of course, experts' time and knowledge are expensive and limited. 
  
  All aforementioned gave rise to the attempts of solving the problem and delivering knowledge by automated machine learning methods. This solution still requires experts' skills, but less than in the case of manual knowledge extraction. One of the promising directions for working on Relation Extraction is \textit{Deep Learning}. But Deep Learning models usually require a lot of training data in order to learn. Taking into account that best manual labels are hard to get the approach of \textit{Distant Supervision} was used in this thesis. Distant Supervision allows using a small amount of gold standard data in order to get a large amount of approximate training data.
  
  The \textit{medical domain} can be seen as an example field. It is very critical to have automated knowledge extractors for publications and articles that might contain the answers to everyday questions in doctoral practice. Taking into account the large volume of research and a few experts in the area, the domain creates a perfect example of the domain that needs automating of the knowledge extraction as developed in this thesis.
  
  The goal of this thesis is to investigate the possibilities of Convolutional Neural Networks together with Distant Supervision and Multiple Instance Learning in solving the problem of Relation Classification. In order to see the power of the approach, it is tested not only in the medical domain but in a general domain as well. The approach of supervised training is compared to Distant Supervision to find out the benefits and drawbacks of the latter. Multi-Instance Learning is considered as an approach that can improve Distant Supervision.
  
  Overall experiments proved the possibility of using existing knowledge and raw textual data for automatically created training dataset that allows training the model to sufficient level without manual labelling. Distantly supervised data added to existing supervised data might improve Recall, that allows experts to concentrate attention only on classified sentences and not to check all additional textual data.}
